%This document wants to the explain the quran package with some examples.
\documentclass{article}

\usepackage{forloop}
\usepackage{quran}
%The xepersian package automatically load bidi, and I've loaded it because I want to set a font that supports Arabic letters
\usepackage{xepersian}

% This macro set the main text font for non-latin letter, and it can scale font.
\settextfont[Scale=1.1]{Scheherazade}

\def\surna[#1]{\centerline{\hss\surahname*[#1]\hss\surahname[#1]\hss}}
\def\test#1{
    \par
    \surna[#1]
    \quransurah*[#1]
    \bigskip
}

\begin{document}

% For typesetting بِسمِ اللَّهِ الرَّحمٰنِ الرَّحيمِ use below macro
%\centerline{\basmalah}

\quransurah[108] % Surah Al-Kauther

%%\surna[110]\quransurah*[110]  % Surah Al-Nasr

%The below typeset 104th surah through 113th surah.
%%\quransurah*[104-113]

\makeatletter
\surna[\qt@surah@default]\quransurah*  % Surah Al-Ikhlas
\makeatother

\quranayah[33][33]
\quranayah*[76][1-22]

%%You can typeset whole of Holy Quran with below commands.
%%\newcounter{ct}
%%\forloop{ct}{1}{\value{ct} < 115} {\test{\value{ct}}}

%%You can typeset whole of Holy Quran with below commands.
%%\newcounter{jz}
%%\forloop{jz}{1}{\value{jz} < 31} {\quranjuz[\value{jz}]}

%%\quranjuz*[28-30]

%\quranpage*[256]
%\quranpage*[3-4]

%%You can typeset whole of Holy Quran with below commands, just replace 8 with 605.
\newcounter{pg}
\forloop{pg}{1}{\value{pg} < 8} {
\hfill  صفحة  \arabic{pg} \par
\quranpage*[\value{pg}]\vfill}

%\quranhizb*[117-120]

%\quranquarter*[1-4]
%\quranquarter*[239-240]

%\quranruku[313]
%\quranruku[556]

%\quranmanzil*[2]


%%\surna[1]\qurantext  % Surah Al-Hamd

%%\surna[1]\qurantext* % Surah Al-Hamd

%%\surna[114]\qurantext[6231-6236]  % Surah Al-Nas

%%\surna[114]\qurantext*[6231-6236]  % Surah Al-Nas

%%%\surna[2]\qurantext*[8-293] % Surah Al-Baqara

%\qurantext[1-6236] % The whole of Holy Quran


\newcount\mysurah
\newcount\myayah
\indexconvert{1436}{\mysurah}{\myayah}
سوره \the\mysurah

آیه  \the\myayah
\end{document}
