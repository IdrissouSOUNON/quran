\documentclass{ltxdoc}
%\usepackage{pgffor}
\usepackage{forloop}
\usepackage{pgf}
%\usepackage{pgfcalendar}
\usepackage{framed}
\usepackage{holtxdoc}
\usepackage{url}
\usepackage{quran}

\begin{document}
\title{The \xpackage{quran} Package\footnote{to one who has devoted his life to Quran}}
\author{Seiied Mohammad Javad Razavian\\\xemail{javadr@gmail.com}}

\date{Apr 21, 2016}
\maketitle

\begin{framed}
This manual is incomplete and the package is under development. \par 
\centerline{ver 1.1}
\end{framed}

\tableofcontents

\section{Introduction}
This package is prepared for typesetting the Holy Quran. This work is inspired by \xpackage{lipsum} and 
\xpackage{ptext} packages, provides several macros for
typesetting the whole or any parts of the Holy Quran based on its popular division.

\section{Loading Package}
You can load the package in the ordinary way
\cs{usepackage[option]\{quran\}}.
%Like the other pakages that use \xpackage{bidi}, it should be loaded just before \xpackage{bidi}.

\section{Typesetting The Holy Quran}
For typesetting any parts or whole of The Holy Quran there are several macros in the package.

\subsection{Surah/Sovar}
\begin{declcs}{quransurah}\oarg{m}
\end{declcs}
This macro can typeset any Surah of Quran. As you know there are 114 Sovar, so \textit{m} could 
be an integer between 1 to 114. The macro also could accept the transliteration of surah's name, 
e.g. both of \cs{quransurah[1]} and \cs{quransurah[Al-Fatiha]} have the same effect. For using 
surah's name instead of their order see table~\ref{tab1}.
\marginpar{\cs{quransurah[1]}}\marginpar{\cs{quransurah[Al-Fatiha]}}

\newcounter{ct}
\def\mycell#1{
    \surahname[\value{#1}]
    \addtocounter{#1}{38}
}
\begin{table}[!htbp]
    \centering
    \fontsize{8}{10}\selectfont
    \begin{tabular}{|*{3}{c|l|}}
    \hline
     order & surahname & order  & surahname  & order  & surahname  \\
    \hline\hline%
    \forloop{ct}{1}{\value{ct} < 38}{%
        \thect & \mycell{ct} & \thect & \mycell{ct} & \thect & \mycell{ct}  \\
        \addtocounter{ct}{-114}
    }%
    \thect & \mycell{ct} & \thect & \mycell{ct} & \thect & \mycell{ct}\\%
    \hline
    \end{tabular}
    \protect\caption{Surah's name trasliteration.}\label{tab1}
\end{table}

\begin{declcs}{quransurah}\oarg{m-n}
\end{declcs}
This acts like above but can typeset \textit{m}$^{th}$ Surah to \textit{n}$^{th}$  one. 

\begin{declcs}{quransurah}\oarg{m{-}{-}n}
\end{declcs}
In this case m and n are tranliterations of surah's names, e.g. \cs{quransurah[An-Nasr--An-Nas]}.
\marginpar{\cs{quransurah[An-Nasr--An-Nas]}}

\begin{declcs}{quransurah}
\end{declcs}
Using \cs{quransurah} without its optional argument typeset the default Surah--Al-Ikhlas.
To change the default Surah use \cs{setsurahdefault\{m\}}, again m must be between 1 and 114; 
package's default is 112.
\marginpar{\cs{setsurahdefault}}

\subsection{Ayah/Ayat}
\begin{declcs}{quranayah}\oarg{l}\oarg{n}
\end{declcs}
The above macro typeset the \textit{n}$^{th}$ Ayah of \textit{l}$^{th}$ Surah. 
If you just typeset the first Ayah of any Surah, then the Basmalah will not be printed.
In this case if you need Basmalah, then you can use the \cs{Basmalah} macro before.
\marginpar{\cs{Basmalah}}

\begin{declcs}{quranayah}\oarg{l}\oarg{m-n}
\end{declcs}
This macro typeset the \textit{m}$^{th}$ to \textit{n}$^{th}$ Ayat of l$^{th}$ Surah. 

\par
\addvspace{4.5ex plus 1ex}%
\vskip -\parskip
\noindent\textit{l} could be an integer less than or equal to 114. It also could be 
the transliteration of  l$^{th}$ Surah's name, see table~\ref{tab1}. 
\textit{n} is diffirent by every Surah and it relies on the number of the Surah's verses. 
\textit{m} should be between 1 and a number less than \textit{n}. 

\subsection{Page/Pages}
\begin{declcs}{quranpage}\oarg{n}
\end{declcs}
Typeset the \textit{n}$^{th}$ page of Quran and the macro below typeset \textit{m}$^{th}$ to 
\textit{n}$^{th}$ pages of Quran. 

\begin{declcs}{quranpage}\oarg{m-n}
\end{declcs}
The numbers should be between $1$ to $604$, based on Uthmani text. 

\subsection{Juz/Ajza}
\begin{declcs}{quranjuz}\oarg{n}
\end{declcs}
Above macro typesets the \textit{n}$^{th}$ juz of Quran and below macro typesets m$^{th}$ to n$^{th}$ juzzes of Quran. 
\begin{declcs}{quranjuz}\oarg{m-n}
\end{declcs}
The numbers should be between $1$ to $30$.

\subsection{Hizb/Ahzab}
Each juz of Quran is devided into two ahzab (group),  so there are 60 ahzab.
Following macros typeset hizb/ahzab of Quran. 

\begin{declcs}{quranhizb}\oarg{n}
\end{declcs}
\begin{declcs}{quranhizb}\oarg{m-n}
\end{declcs}

\subsection{Quarter/Quarters}
Eash hizb of Quran is devided into four quarters, making eight quarters per juz, called maqra. 
There are 240 of these quarters (of hizb) in the Quran. These maqra are often used as sections for revision when memorizing the Quran.\footnote{\url{https://en.wikipedia.org/wiki/Juz'}}
\begin{declcs}{quranquarter}\oarg{n}
\end{declcs}
\begin{declcs}{quranquarter}\oarg{m-n}
\end{declcs}

Above macros typeset the \textit{n}$^{th}$ or m$^{th}$ to n$^{th}$ quarter(s) of Quran.   

\subsection{Ruku/Rukus}
``The term ruk\= uʿ --- roughly translated to ``passage", ``pericope" or ``stanza" --- is also used to denote a group of thematically related verses in the Qur'an. Longer suras (chapters) in the Qur'an are usually divided into several ruk\= us, so that the reciters could identify when to make ruk\= u in Salat without breaking an ongoing topic in the Quranic text."\footnote{\url{https://en.wikipedia.org/wiki/Ruku}}

The following macros can typeset rukus of The Holy Quran the way the above macros act. 

\begin{declcs}{quranruku}\oarg{n}
\end{declcs}
\begin{declcs}{quranruku}\oarg{m-n}
\end{declcs}
The numbers should be between $1$ to $556$.

\subsection{Manzil/Manazil}
``For the convenience of people who wish to read the Qur'an in a week the text may be divided into 7 portions, each portion is known as Manzil.

The following division to 7 equal portions is by Hamza Al-Zayyat (d.156/772):
\begin{enumerate}
    \item Al-Fatihah (chapter 1) through An-Nisa' (chapter 4) consisting of 4 surahs.
    \item Al-Ma'ida (chapter 5) through At-Tawba (chapter 9) consisting of 5 surahs.
    \item Yunus (chapter 10) through An-Nahl (chapter 16) consisting of 7 surahs.
    \item Al Isra' (chapter 17) through Al-Furqan (chapter 25) consisting of 9 surahs.
    \item Ash-Shuara' (chapter 26) through Ya-Seen (chapter 36) consisting of 11 surahs.
    \item As-Saaffat (chapter 37) through Al-Hujarat (chapter 49) consisting of 13 surahs.
    \item Qaf (chapter 50) through An-Nass (chapter 114) consisting of 65 surahs.''\footnote{\url{https://en.wikipedia.org/wiki/Manzil}}
\end{enumerate}

Following macros typeset manzil/manazil of Quran.

\begin{declcs}{quranmanzil}\oarg{n}
\end{declcs}
\begin{declcs}{quranmanzil}\oarg{m-n}
\end{declcs}

\subsection{Text of Quran}
The following macro is the heart of all macros that can typeset any range of Quran. As you know there 
are $6236$ Ayat in Quran. This macro can typeset a specific Ayah or any range of Ayat. 

\begin{declcs}{qurantext}\oarg{n}
\end{declcs}
Typesets the \textit{n}$^{th}$ Ayah of Quran. 

\begin{declcs}{qurantext}\oarg{m-n}
\end{declcs}
Typesets any range of Ayat from the \textit{m}$^{th}$ to \textit{n}$^{th}$ Ayah. 
 
Using \cs{qurantext} without its optional argument typeset [1-7] --Al-Hamd. To
change the default text use \cs{setqurantextdefault\{m-n\}}
\marginpar{\cs{qurantext}}
\marginpar{\cs{setqurantextdefault\{m-n\}}}

\subsection{Whole of The Holy Quran}
The following macros can typeset whole of The Holy Quran:

\begin{itemize}
    \item \cs{quransurah[1-114]}
    \item \cs{quranjuz[1-30]}
    \item \cs{quranpage[1-604]}
    \item \cs{qurantext[1-6236]}
    \item \cs{quranhizb[1-60]}
    \item \cs{quranquarter[1-240]}
    \item \cs{quranruku[1-556]}
    \item \cs{quranmanzil[1-7]}
\end{itemize}

\section{Miscellaneous}

\subsection{Name of Surah}
\begin{declcs}{surahname}\oarg{m}
\end{declcs}
This macro returns the tranliteration of \textit{m}$^{th}$ Surah.

\begin{declcs}{surahname*}\oarg{m}
\end{declcs}
This macro returns the arabic name of \textit{m}$^{th}$ Surah.

\subsection{Basmalah}
\begin{declcs}{basmalah}
\end{declcs}
It provides the basmalah in arabic text which every Surah is started by that.

\subsection{Index Converting}
\begin{declcs}{indexconvert}\marg{index}\marg{surah macro}\marg{ayah macro}
\end{declcs}
This command converts an index number between 1 to 6236 
to its exact surah's index and ayah's index. The index must be a number or
\TeX{} counter, the surah macro and ayah macro must be \TeX{} macro names. They
will be set to numbers representing the surah and ayah of the given index in whole of Quran.

\section{Options to The Package}
All text generated by macros of this package will be separated by the macro \cs{par}. Sometimes, 
a version of text without \cs{par} is desired. Therefore the package provides the option 
\texttt{nopar}\marginpar{\texttt{nopar}}
that causes no macros of the package terminated by \cs{par}.
Another way for achieving this, is using the starred version of macros. All macros of the package
has a starred version that omit the \cs{par} macro in typesetting the text of Quran. 

In some cases maybe user wants to typeset the Ayat without their following numbers that can be seen in printed version of The Holy Quran. The \texttt{nonumber}\marginpar{nonumber} 
option tells the package that print no number after ayah.
But sometimes we want to typeset just some Ayat without theirs, so there is a \cs{ToggleAyahNumber} macro
that can cause numbers will be omitted. With \texttt{nonumber} option this macro acts vice versa, it mean it 
causes that Ayat number will be printed.

\section{Acknowledgment}
At first, I should appreciate Vafa Khaligi by preparing such adorable packages --\xpackage{bidi} and 
\xpackage{xepersian}. Secondly, the costly \url{http://tanzil.net/} for providing the text of The Holy Quran that 
is developed by Hamid Zarrabi-Zadeh. And finally, people in Parsi\LaTeX\footnote{\url{http://parsilatex.com/}}
group who tested and commented this work in advance. 

\end{document}
